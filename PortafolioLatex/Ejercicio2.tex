\documentclass{article}
\usepackage{amsmath}

\begin{document}

\title{Sumas de Riemann}
\author{Tú Nombre}
\date{\today}
\maketitle

Las sumas de Riemann son una fosrma de aproximar el área bajo una curva mediante la partición del dominio en subintervalos y la suma de áreas de rectángulos.

\section{Sumas de Riemann Izquierda askdsadksas}

La suma de Riemann izquierda parasa una función $f(x)$ en el intervalo $[a, b]$ con $n$ subintervalos se define como:

\[
S_L = \sum_{i=1}^{n} f(x_{i-1}) \Delta x
\]

donde $\Delta x = \frac{b-a}{n}$ y $x_i = a + i \Delta x$.

\section{Sumas de Riemann Derechas}

La suma de Riemann derecha para la misma función y partición se define como:

\[
S_R = \sum_{i=1}^{n} f(x_i) \Delta x
\]

\section{Sumas de Riemann Media}

La suma de Riemann media utiliza el valor promedio de la función en cada subintervalo:

\[
S_M = \sum_{i=1}^{n} f\left(\frac{x_{i-1}+x_i}{2}\right) \Delta x
\]

Estas sumas de Riemann proporcionan aproximaciones del área bajo la curva de $f(x)$ en el intervalo $[a, b]$.

\end{document}